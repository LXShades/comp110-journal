\documentclass{journal}

\title{On Goto Killing You in Your Sleep}
\author{1707981}
\date{\today}

\begin{filecontents*}{references.bib}
@article{goto,
 author = {Dijkstra, Edsger W.},
 title = {Letters to the Editor: Go to Statement Considered Harmful},
 journal = {Commun. ACM},
 issue_date = {March 1968},
 volume = {11},
 number = {3},
 month = mar,
 year = {1968},
 issn = {0001-0782},
 pages = {147--148},
 numpages = {2},
 url = {http://doi.acm.org.ezproxy.falmouth.ac.uk/10.1145/362929.362947},
 doi = {10.1145/362929.362947},
 acmid = {362947},
 publisher = {ACM},
 address = {New York, NY, USA},
 keywords = {alternative clause, branch instruction, conditional clause, go to statement, jump instruction, program intelligibility, program sequencing, repetitive clause},
} 

@article{algolwirth,
 author = {Wirth, Niklaus and Hoare, C. A. R.},
 title = {A Contribution to the Development of ALGOL},
 journal = {Commun. ACM},
 issue_date = {June 1966},
 volume = {9},
 number = {6},
 month = jun,
 year = {1966},
 issn = {0001-0782},
 pages = {413--432},
 numpages = {20},
 url = {http://doi.acm.org.ezproxy.falmouth.ac.uk/10.1145/365696.365702},
 doi = {10.1145/365696.365702},
 acmid = {365702},
 publisher = {ACM},
 address = {New York, NY, USA},
} 

@article{algolguide,
	title = "An introduction to ALGOL 60",
	author = "H. Bottenbruch",
	year = "1962",
	pages = "161-221",
	volume = "9",
	issue = "2",
	address = "New York, NY, USA",
	journal = "Journal of the ACM (JACM)"
}

@article{gotoodeep,
	title = "Go to considered harmful considered harmful",
	journal = "Commun. ACM",
	author = "F. Rubin",
	pages = "195-196",
	year = "1987",
	month = mar,
	volume = "30",
	issue = "3",
}

@misc{disappointing,
	title = "On a somewhat disappointing correspondence (EWD1009)",
	author = "Edsger W. Dijkstra",
	year = "1987",
	month = may,
	url = "http://www.cs.utexas.edu/users/EWD/ewd10xx/EWD1009.PDF"
}

@article{loopbreaking,
 author = {Bochmann, G. V.},
 title = {Multiple Exits from a Loop Without the GOTO},
 journal = {Commun. ACM},
 issue_date = {July 1973},
 volume = {16},
 number = {7},
 month = jul,
 year = {1973},
 issn = {0001-0782},
 pages = {443--444},
 numpages = {2},
 url = {http://doi.acm.org.ezproxy.falmouth.ac.uk/10.1145/362280.362300},
 doi = {10.1145/362280.362300},
 acmid = {362300},
 publisher = {ACM},
 address = {New York, NY, USA},
 keywords = {control structures, exit statement, goto free programming, multiple exits from loops},
} 

@inproceedings{gotostudy,
 author = {Nagappan, Meiyappan and Robbes, Romain and Kamei, Yasutaka and Tanter, \'{E}ric and McIntosh, Shane and Mockus, Audris and Hassan, Ahmed E.},
 title = {An Empirical Study of Goto in C Code from GitHub Repositories},
 booktitle = {Proceedings of the 2015 10th Joint Meeting on Foundations of Software Engineering},
 series = {ESEC/FSE 2015},
 year = {2015},
 isbn = {978-1-4503-3675-8},
 location = {Bergamo, Italy},
 pages = {404--414},
 numpages = {11},
 url = {http://doi.acm.org.ezproxy.falmouth.ac.uk/10.1145/2786805.2786834},
 doi = {10.1145/2786805.2786834},
 acmid = {2786834},
 publisher = {ACM},
 address = {New York, NY, USA},
 keywords = {Dijkstra, Empirical SE, Github, Use of goto statements},
} 

@article{againstgoto,
 author = {Wulf, William A.},
 title = {A Case Against the GOTO},
 journal = {SIGPLAN Not.},
 issue_date = {November 1972},
 volume = {7},
 number = {11},
 month = nov,
 year = {1972},
 issn = {0362-1340},
 pages = {63--69},
 numpages = {7},
 url = {http://doi.acm.org.ezproxy.falmouth.ac.uk/10.1145/987361.987372},
 doi = {10.1145/987361.987372},
 acmid = {987372},
 publisher = {ACM},
 address = {New York, NY, USA},
 keywords = {goto-less programming, programming, programming languages, structured programming},
}

@article{pascal,
	title = "The Programming Language Pascal",
	author = "N. Wirth",
	month = "Mar",
	year = "1971",
	volume = "1",
	journal = "Acta Informatica",
	pages = "35-63",
	address = "Secaucus, NJ",
	url = "https://amaus.net/static//S100/software/pascal/1970%20The%20pascal%20programming%20language.pdf"
}

@incollection{pascalrecollections,
 author = {Wirth, N.},
 chapter = {Recollections About the Development of Pascal},
 title = {History of Programming languages---II},
 editor = {Bergin,Jr., Thomas J. and Gibson,Jr., Richard G.},
 year = {1996},
 isbn = {0-201-89502-1},
 pages = {97--120},
 numpages = {24},
 url = {http://doi.acm.org.ezproxy.falmouth.ac.uk/10.1145/234286.1057812},
 doi = {10.1145/234286.1057812},
 acmid = {1057812},
 publisher = {ACM},
 address = {New York, NY, USA},
} 

@article{pathfinding,
 author = {Dijkstra, E. W.},
 title = {A Note on Two Problems in Connexion with Graphs},
 journal = {Numer. Math.},
 issue_date = {December  1959},
 volume = {1},
 number = {1},
 month = dec,
 year = {1959},
 issn = {0029-599X},
 pages = {269--271},
 numpages = {3},
 url = {http://dx.doi.org.ezproxy.falmouth.ac.uk/10.1007/BF01386390},
 doi = {10.1007/BF01386390},
 acmid = {2722945},
 publisher = {Springer-Verlag New York, Inc.},
 address = {Secaucus, NJ, USA},
} 

@article{truths,
 author = {Dijkstra, Edsger W.},
 title = {How Do We Tell Truths That Might Hurt?},
 journal = {SIGPLAN Not.},
 issue_date = {May 1982},
 volume = {17},
 number = {5},
 month = may,
 year = {1982},
 issn = {0362-1340},
 pages = {13--15},
 numpages = {3},
 url = {http://doi.acm.org.ezproxy.falmouth.ac.uk/10.1145/947923.947924},
 doi = {10.1145/947923.947924},
 acmid = {947924},
 publisher = {ACM},
 address = {New York, NY, USA},
} 

@book{structured,
 editor = {Dahl, O. J. and Dijkstra, E. W. and Hoare, C. A. R.},
 title = {Structured Programming},
 year = {1972},
 isbn = {0-12-200550-3},
 source = {Library of Congress Catalog Card Number: 72-84452},
 publisher = {Academic Press Ltd.},
 address = {London, UK, UK},
} 

@book{cprogramming,
	title = "The C Programming Language",
	author = "Brian W. Kernighan and Dennis M. Ritchie",
	year = "1978",
	address = "Upper Saddle River, NJ",
	url = "http://www.dipmat.univpm.it/~demeio/public/the_c_programming_language_2.pdf"
}

@inproceedings{github,
 author = {Kalliamvakou, Eirini and Gousios, Georgios and Blincoe, Kelly and Singer, Leif and German, Daniel M. and Damian, Daniela},
 title = {The Promises and Perils of Mining GitHub},
 booktitle = {Proceedings of the 11th Working Conference on Mining Software Repositories},
 series = {MSR 2014},
 year = {2014},
 isbn = {978-1-4503-2863-0},
 location = {Hyderabad, India},
 pages = {92--101},
 numpages = {10},
 url = {http://doi.acm.org.ezproxy.falmouth.ac.uk/10.1145/2597073.2597074},
 doi = {10.1145/2597073.2597074},
 acmid = {2597074},
 publisher = {ACM},
 address = {New York, NY, USA},
 keywords = {Mining software repositories, bias, code reviews, git, github},
} 
	
@article{structuredgoto,
 author = {Knuth, Donald E.},
 title = {Structured Programming with Go to Statements},
 journal = {ACM Comput. Surv.},
 issue_date = {Dec. 1974},
 volume = {6},
 number = {4},
 month = dec,
 year = {1974},
 issn = {0360-0300},
 pages = {261--301},
 numpages = {41},
 url = {http://doi.acm.org.ezproxy.falmouth.ac.uk/10.1145/356635.356640},
 doi = {10.1145/356635.356640},
 acmid = {356640},
 publisher = {ACM},
 address = {New York, NY, USA},
} 

@inproceedings{humanprogramming,
	title = "Programming considered as a human activity",
	author = "Edsger W. Dijkstra",
	journal = "Proceedings of the IFIP Congress 1965",
	pages = "213-217",
	month = may,
	year = "1965"
}

@book{csharp,
	title = "The C\# Programming Language",
	edition = "4",
	author = "Anders Heljsberg et al.",
	month = oct,
	year = "2010",
	url = "http://it.guldstadsgymnasiet.se/c%23/C%23%20Programming%20Language,%20The,%204th%20Edition.pdf"
}

@article{revolution1,
 author = {Lecarme, Olivier},
 title = {Structured Programming, Programming Teaching and the Language Pascal},
 journal = {SIGPLAN Not.},
 issue_date = {July 1974},
 volume = {9},
 number = {7},
 month = jul,
 year = {1974},
 issn = {0362-1340},
 pages = {15--21},
 numpages = {7},
 url = {http://doi.acm.org.ezproxy.falmouth.ac.uk/10.1145/953224.953226},
 doi = {10.1145/953224.953226},
 acmid = {953226},
 publisher = {ACM},
 address = {New York, NY, USA},
} 



\end{filecontents*}

\usepackage{graphicx}
\usepackage{xcolor}
\usepackage{listings}

\definecolor{codebackground}{rgb}{0.8, 0.8, 0.8}

\lstdefinestyle{genius}{
	tabsize = 2,
	basicstyle = \tiny,
    captionpos = b,
    backgroundcolor = \color{codebackground},
    breaklines = true,
    frame = single
}

\lstset{style=genius}

\begin{document}
\maketitle

% Basic competency: 40% 1 peer review, viva discussion, citations appropriate, academically integral
% Breadth of reading: 10% -- read lots of articles, 16 of them!
% Depth of insight: 15% Significant insight demonstrated. Discussion is anlytical and evaluative in nature
% Specificability, verifiablility and accuracy of claims: 10%, all claims have a clear source of evidence. Almost no errors/misinterpretations.
% Synthesis: 15%, Info from multiple sources synthesised into a strongly cohesive whole. Connections are analytical and evaluative.
% Spelling & grammar: 5%
% Structure: 5%
% \textbf \textit \underline

% https://dl-acm-org.ezproxy.falmouth.ac.uk/citation.cfm?id=363570&CFID=828610973&CFTOKEN=85313317
% response to Dijkstra: "How many poor, innocent, novice programmers will feel guilty when their sinful use of go to is flailed in this letter?"

% The humble programmer
% In his marriage certificate Dijkstra is a 'theoretical physicist'

\section{\textunderscore \textunderscore Introduction:;}
In March 1968, the notorious 'Go to statement considered harmful' \cite{goto} letter was penned by Edsger W. Dijkstra, an influential pioneer of computer science during its early days -- particularly if `influential' were a measurement of the vast swathes of `considered harmful' papers one could instantly inspire across acedemia (CITE), plus a revolution for structured programming pushing exhaustive efforts to deprecate the goto statement \cite{revolution1, againstgoto}. His paper can be summarised as a criticism that using goto statements causes the flow of the program's execution to diverge greatly from the flow of text, by allowing the process to jump to various areas in code in an unstructured manner, with no obvious indicators from where it came. Due to the compromise in human readability he proposed that the use of goto statements be avoided outside machine level code.

Dijkstra's words were bold on their cricitism of what was once common practice in the era[CITE]. While his motivation was apparently just a negative correlation between a programmer's skill quality and the number of goto statements in their code \cite{goto, humanprogramming}, the observent would recognise Dijkstra as the creator of the popular pathfinding algorithm now known by his name \cite{pathfinding}, and by its complexity could conclude that his objection to the goto statement was motivated because the ability to simply `go to' some code is far too easy. Dijkstra's subsequent collision-free descent into madness, that pulled many programmers and academics along with him in the cold grasps of the ``considered harmful'' phenomenon, in the ever deeping existential crisis illustrating how virtually everything in programming is flawed in some way as our souls are meanwhile sold to the megacorporations of tomorrow \cite{truths}, began around here.

For the next part \textbf{goto ``To Pascal''}

\section{\textunderscore \textunderscore Conclusion:;}
If you arrived here before reading the rest of the essay please restart the program.

Today it is truly a wonder whether the goto statement has a right to exist, similar to the enigma of whether programmers have a right to exist with it. This appears to be why the goto statement is still around today; while perhaps only 10\% of uses \cite{github} go backwards, nobody can truly predict how the volatile community of programmers would react to a stripped feature. Thus it seems that goto is here to stay, at least until the next revolutionary programming language comes along. With the dawn of quantum computing, there is hope yet; until then, we accept our traditional programming languages standing at around 30 to 50 years old [CITE], barely escaping the goto controversy in their days, and thus hesitate to `go to' anywhere new anytime soon.

\section{\textunderscore \textunderscore Goto\textunderscore in\textunderscore Practice:;}
To see whether a backwards goto statement should be useful, it is worth discovering where goto is typically used in the first place. An extensive study \cite{gotostudy} of the world's largest open code repository \cite{github}, GitHub, found that most goto statements in modern C code are used for error handling. This is usually characterised by a function, sometimes one which loads various error-dependent resources, followed by a function end returning success, then followed by an `error:' label of such, beneath which any loaded resources are freed before returning. This is popular, but does not necessitate a backwards-capable goto statement as the error condition is usually placed at the end of the function \cite{gotostudy}.

\begin{lstlisting}[language=C++,caption={An example of goto used for error handling (C++)}]
bool DeliberatelyCorruptThisEssay() {
	FILE *file1 = NULL, *file2 = NULL, *file3 = NULL;
	
	file1 = fopen("essay.tex", "w");
	[...] // do malicious stuff
	file2 = fopen("excuse_for_no_submission.txt", "w")
	[...] // do suspicious stuff
	if (!file2)
		goto Error; // don't worry I got this brah
	file3 = fopen("allies_in_crime.txt", "w");
	if (!file3)
		goto Error; // goto power corruption intensifies
	[...] // do duplicitous stuff
	Success:
	return true; // essay corrupted and defense constructed
	Error:
	[...] // cleanup for any number of unclosed files
	return false; // uh oh, guess we need to hand this in
}
\end{lstlisting}

The study similarly concluded that 90\% of goto uses were forward only, and believed to be used for reasonable purposes. It can thus be safely assumed that the programmer may in fact have a grasp of the devil's powers after all, and that goto can be used for good, and it is perhaps us who are evil.

It is likely this is the reason goto still exists: it is useful, and seemingly, the bigger concern was never the statement itself, but its habitual use: that was the inspiration behind Dijkstra's paper \cite{goto}. Indeed [CITE] it is clear that . It has however nurtured a culture of avoiding goto at all costs, even in potentially useful scenarios. [CITE]

\section{OOP considered harmful?}

\section{\textunderscore \textunderscore To_Pascal:;}
Dijkstra's disdain was not unique. His paper itself found inspiration from the words of N. Wirth, who with C. A. R. Hoare \cite{algolwirth} advised the demotion of labels and goto statements whilst proposing changes to the ALGOL language in 1965, with a similar rationale that the structure of a program would be clearer with structured 'case' statements and loops. In their proposal, 'goto' would be stripped of all complexities, leaving no more than a functional jump statement should the ability to jump anywhere in code be ever needed.

For a trivial side remark about the proposed case statements, set \textbf{x} to here, then \textbf{goto} "Switch statements".

It is arguable that it was Wirth who took the more active role in demoting goto in favour of structured programming. When he and his team's proposals for ALGOL were rejected in favour of a more complex overhaul (to become ALGOL 68) \cite{pascalrecollections}, Wirth went on to produce ALGOL W, whilst the Working Group created ALGOL 68. Soon after, ALGOL 68 failed, creating a new hole in the market through which Wirth threaded in 1970 with his new language, Pascal \cite{pascalrecollections}.

Wirth's goal in creating Pascal was to advance programming languages -- despite the widespread adoption of far more popular, yet stagnated ones -- resulting in a language with improved structure and an easier learning curve \cite{pascal}. He focussed on encouraging structured programming, explained in great detail in a book partly authored by Dijkstra himself \cite{structured}, wherein code is written in a tree structure, flowing downward roughly proportionally to the internal process.

Yet contrary to all those developments, for some reason, goto statements still existed, even implemented into Pascal itself \cite{pascal}. \textbf{goto ``Prevalence of the goto feature''}.

\section{\textunderscore \textunderscore Prevalence of the goto feature:;}
Like a forbidden fruit or a cigarette box, goto somehow still exists in programming languages, with a large 'DO NOT USE' written on its packet.  In fact, one of the most popular programming languages of today, C, has a functional goto statement, described by D. Ritchie as infinitely abusable and unnecessary \cite{cprogramming} as he gleefully implemented it into the compiler. It calls to question, if goto were truly harmful, why is it still included in modern programming languages? Even Microsoft's modern interpretation of C++, C\#, features a version of goto that is in fact expanded to also support switch labels \cite{csharp}. Why plant evil in the garden, water it and call it a weed?

There are valid uses for goto, some of which were illustrated in F. Rubin's article \cite{gotoodeep}, disputing Dijkstra's original paper (and considered disappointing \cite{disappointing}). A popular use is when breaking out of deeply embedded loops under a particular condition \cite{gotoodeep}. This is in fact also described in the C programming book itself \cite{cprogramming}, raising the question; hypothetically speaking, is there not an alternative way to achieve this? When the issue is that we need to jump to a different area of code, namely a different scope, a jumping statement seems like a reasonable solution. And in order to break out of a scope into a different particular one, would it not be best to refer to it by a name--a label--so that the level of scope can change in the future? Is that not a goto statement?

\begin{lstlisting}[language=Python,caption={An example of goto used for breaking loops (C)}]
void ImInTooDeep() {
	for (Friend* f = friends[0]; f < &friends[numFriends]; f++) {
		for (Friend* ff = f->friends[0]; ff < &f->friends[f->numFriends]; ff++) {
			for (Friend* fff = ff->friends[0]; fff < &ff->friends[ff->numFriends]; ff++) {
				goto FoundSomeone; // no other way to break all loops at once
			}
		}
	}
	
	printf("No-one relevant to your life likes your Pikachu plushie\n");
	FoundSomeone:
	printf("There is a glimmer of hope. Pika!");
	// ...but then they never returned. for this was a void function
}
\end{lstlisting}

An article by W. Wulf \cite{againstgoto} addresses one particular solution to the above problem, featured in the Bliss programming language: to include a 'leave' statement to escape from a loop and jump to a label. This is safer than a goto statement with its risk of making code unreadable (which is goto's fault [CITE MANY], and definitely not the programmer's), since it can only jump forward. This is a very logical solution, but calls to question whether it is really worthy to replace goto, as merely a restricted form of the same thing? A deeper question would be whether there is any use left for a goto statement that goes backwards in a world wherein many believe a backwards goto in any text is explicitly evil[CITE]?

To find out \textbf{goto ``Goto in practice;'}

It may be that `goto' is included just to ensure the feature is available for special cases. However, given the extreme unpopularity of the goto statement[CITE], programmers are likely to avoid using it, sometimes at the expense of line count and computation time.

\section{\textunderscore \textunderscore Switch statements:;}
Interestingly, the 'case' statement proposal was intended to replace the 'switch' statement of ALGOL 60, which at the time was effectively a declaration of an array of labels to be used with a dynamic 'goto' statement \cite{algolguide}. What makes this interesting is the contrast to the modern switch statement of the C family today, which essentially defines 'switch' as what was once Pascal's 'case', and 'case' as what was once the labels, but restricted within the scope of the switch statement. As C was inspired by the ALGOL family, including Pascal, it is likely that the creator of C considered this small reinterpretation as a more correct terminology.

[put a figure here illustrating the different uses of the switch statement]

\bibliography{references}
\bibliographystyle{ieeetr}
\end{document}